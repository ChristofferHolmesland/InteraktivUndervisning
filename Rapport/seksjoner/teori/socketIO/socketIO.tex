\subsection{Socket.IO}
Socket.IO er den websocketen som er mest brukt til Node.JS. Noe som gjør at det også er den som er 
mest støttet og den hvor en kan få mest hjelp når en sitter fast. Socket.IO bygger på ENGINE.IO og gjør at en
kan få en link mellom server og klient i santid, som fungerer begge veier og bygger på en hendelse basert 
kommunikasjon. I denne web applikasjonen vil flere brukere sende oppdateringer til serveren og serveren 
vil sende ut oppdateringer til flere klienter samtidig. Dette hadde blitt veldig tungvidt og komplisert
å få til med å sende AJAX eller HTTP forespørsler. Når en Socket.IO forbinelse har blitt satt opp så
er det lett å sette opp funksjonene slik at det er kun den informasjonen som klienten skal ha som blir sendt.
Socket.IO har også en funksjon som heter "room" som gjør at en kan legge en forbindelse inn i et virtuelt rom.
Da kan serveren legge alle klientene som er med i en quiz inn i et bestemt rom og sende oppdateringer til de 
som er med i det rommet. Dette gjør at det også er mulig å skalere serveren opp til å håndtere flere quizer til
å kjøre samtidig. Alle Socket.IO hendelse funksjone er også asynkrone noe som gjør at klienter trenger ikke å
vente på svar lengre enn det serveren bruker på å finne den informasjonen en trenger.