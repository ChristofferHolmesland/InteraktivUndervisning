\subsection{Mocha}
When writing software bugs are often introduced without the developer noticing. As a program grows it gets very time consuming to test every possible execution path manually. To solve this tests are made. There are three different types of tests, unit testing, integration testing, and system testing. Unit tests are used to test smaller pieces of code which doesn't rely on other parts of the application. Integration tests are made to test if different subsystems of the application work togheter in the expected way. System tests are made to test if the system as a whole works as inteded. Mocha is a testing framework which provides good support for testing asynchronous code. Because web development is often asynchronous, Mocha was chosen for writing unit and integration tests. System tests can't be written in Mocha because there is no way to open a browser and simulate user interaction. Other options which was considered are QUnit and Jasmine. Both of them doesn't provide the same support for testing asynchronous code as Mocha, and were therefore not selected.