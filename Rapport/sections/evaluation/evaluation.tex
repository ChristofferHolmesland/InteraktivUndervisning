\section{Evaluation}
\subsection{Vue}
To make the interface between the code and the website simpler, we wanted to use a web framework. We needed a dynamic webpage where the different parts were constantly changing depending on the user interaction. Code separation was important for us because we wanted to be able to split it up as much as possible to minify the risk of one error affecting the entire application. We expected Vue to make this easier. Code reuse was also important. We knew from the beginning that the questions and solutions were going to be displayed on different pages, and we wanted only to write the code for this once. We had previously used templating engines, which makes it possible to use JavaScript variables directly in the HTML. This is something we expected Vue to do for us.
\\[11pt]
We had no experience in making single page applications, so we had a lot to learn. Vue met our expectations, and also had many features we did not know about. Examples are the Vue Router. We did not consider the fact that when there is only one page, the user cannot use the URL bar of the browser to navigate. We had some problems with storing application state between different views and components, but after some searching, we found Vuex which was made to solve that problem. Our experience is that Vue did not prevent us from doing anything we wanted to do, and it provided good solutions to common problems. Much time was spent in the beginning trying to learn Vue, but the time investment was worth it considering how much time we would have spent on recreating the same features found in Vue.

\subsection{Express}
From before, we had experience using Express and knew it supported the use of middleware functions. We assumed this would make the process of implementing OAuth2/OpenID Connect easier. Based on research, we expected Express to work well with Socket.IO, SQLite and Vue. Based on our experience during the development, this turned out to be true even though some configuration had to be done.

\subsection{Socket.IO}
In the web programming DAT310 course, we were taught AJAX requests. Even though this works well when the client needs information from the server, there is a limitation when the server needs to send a message to a client. During the software development DAT210 course, one of our group members worked with WebSockets. WebSockets allowed there to be a two-way connection between the server and clients. This seemed like a good solution for our project since the server needed to talk to clients when running a session. Socket.IO is the only well-developed WebSocket for a JavaScript-based server; therefore it was a natural choice. Despite being the only well-developed option, it fulfilled our needs and expectations.

\subsection{SQLite}
Our experience is that SQLite works as well as other database management systems. It was advantageous not to have to worry about a database server while developing the application. However, it surprised us to learn that foreign keys constraints were not enforced by default. Towards the end of development, we also discovered that SQLite only supports 60 transactions per second on an average HDD. If the limit is reached, the transactions build up in a queue and is ran once the database is free. After looking into the issue, we discovered our server requires following amount of transaction per action:
\begin{center}
    \begin{tabular}{|l|c|}
        \hline
        Action & Transactions \\
        \hline
        Login anonymous user & 0 \\
        \hline
        Login Feide user & 1 \\
        \hline
        Create/Edit question without images & 1 \\
        \hline
        Create/Edit question with images & 2 \\
        \hline
        Create session & 2 \\
        \hline
        Edit/Delete session & 1 \\
        \hline
        Add question to session & 1 \\
        \hline
        Delete user data & 1 \\
        \hline
        Starting/Ending session & 1 \\
        \hline
        End of question round & 1 \\
        \hline
    \end{tabular}
    \captionof{table}{Transactions per action}
    \label{table:transactionsPerAction}
\end{center}
\noindent
Looking at the table \ref{table:transactionsPerAction}, one can see that the limit of 60 transactions can be reached quite quickly. If we had known this from the beginning, we would most likely have chosen a different database management system. Even after taking the last issue into consideration we still think that SQLite can handle the traffic that is produced at the scale it is going to be used at. 
