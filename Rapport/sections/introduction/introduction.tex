\section{Introduction}
\subsection{Task}
The task for the bachelor thesis consisted of developing a game-based application that was to be primarily used in the Algorithms and Data structures. Using the application the instructor would allow the students to play games consisting of questions relevant for the topics introduced in the course.
\\[11pt]
The application was to take inspiration from Kahoot's(insert cite here) interactive game system, were the questions are displayed on a projector and students will answer the questions using their mobile devices. One key difference between application and Kahoot's was that the application had to support a more variety of questions. The user should be able to not only answer the question through multi choice and inputting an answer, but also enable them to interact with some of the data structures taught during in Algorithms and Data structures. This includes inputting data into arrays and stack, and drawing tree and graphs. The former required a basic drawing tool to also be implemented for the application.
\\[11pt] 
The results of the game session was to be stored for the instructor. The instructor can then use the data in order to clear up common mistakes, but used in order improve upon future lectures.
\\[11pt]
The primary target for the application was mobile devices, however it was decided that the application would be more suited as a web application supporting use off both computer and mobile devices. 

\subsection{Goals}
Introduce the chosen goals for the project.
%Have to decide on some common goals based upon the tasks given to use by the bachelor thesis. Whether or not we use the list structure or not is dependent on these goals.
\subsubsection{Main Goals}
\begin{itemize}
\item Goal 1
\item Goal 2
\item Goal 3
\end{itemize}
\subsubsection{Sub Goals}
\begin{itemize}
\item Allow students to store their data using their Feide user as identifier.
\item SubGoal 1 
\end{itemize}

\subsection{Motivation}
The motivation for this application is to give students the opportunity to not only learn about algorithms and data structures through watching the instructor manually implementing all the structures. Doing exercise will allow the students some time to think about what they learned during the lecture. This also allow the students to test themselves of whether or not they've actually learned anything from the lecture. The result after each session is given to the instructor, which in turn can be used for instructor to amend future lectures, but also clear any misconceptions they may have. In sense the application gives both the student and the instructor reliable feedback involving individual and class performance respectively.
\\[11pt]
Other game-based learning applications like Kahoot has seen a lot of use in schools and universities, and for good reason since it helps student interact more during the lecture. However, Kahoot has some flaws that were addressed early on and the application is designed in a way to improve upon or prevent these flaws. For instance, Kahoot has too much of a focus on being a competition rather than self evaluation, that a lot of students can get discouraged from participating or in general being afraid of making mistakes. Kahoot also encourage the players to answer quickly, this can lead to a lot of students answering the first thing that comes to mind, instead of taking their time with each question.
\\[11pt]
Lastly Kahoot is only limited to multi-choice questions, which are too limiting in how we formulate the questions. In algorithms and data structures having only multi-choice questions is too limiting, and instead 'we' have focused on given the instructor ample of ways to structure the questions. This is done so that the instructor can easily choose a response type depending on what the instructor believes will give the best learning experience for participants.

\subsection{Workflow}
Give a summary for technical tools used, and the decision for why they were used in this project. vue.js for templating, server side with js, why js?: node.js, most supported?
