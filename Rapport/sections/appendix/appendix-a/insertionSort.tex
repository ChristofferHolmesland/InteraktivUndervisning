\subsection{Insertion Sort}
Insertion sort is simplest sorting algorithm learned in the algorithms and data structures course. Insertion sort uses an easy algorithm that is forced to go though every entry in a list. The algorithm checks whether or not the current entry has a smaller value then the previous entry, if entry has a smaller value, then the previous entries have to re-sorted with the entry until either you find an entry that is smaller or reach the start of the list. 
\\[11pt]
The insertion sort is quite easy to implement, essentially only consists of a for-loop and a while-loop. Because its simple design its a sorting algorithm that is quite often used. Insertion sort is however quite slow compared to most other sorting algorithms. It only has an O(n) average run time, and a runtime of O($n^2$) in terms of worst case scenario.
\\[11pt]
Although insertion sort is quite effective when it is used to sort almost fully sorted list's. This means that insertion sort algorithm is used by a lot of other sorting algorithms. Mostly used as a final step in these other sorting algorithms. Examples of this being the sorting algorithms Shell sort, Merge sort and Quick sort.\cite{InsertionSort:Geeks,InsertionSort:StudyTonight}	%Probably have should have a cite to the different sorting algorithms mentioned just now.