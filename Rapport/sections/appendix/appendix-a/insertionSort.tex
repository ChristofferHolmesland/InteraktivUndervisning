\subsection{Insertion Sort}
Insertion sort is the simplest sorting algorithm taught in the algorithms and data structures course. Insertion sort uses a simple algorithm that has to go through every entry in a list. The algorithm checks whether or not the current entry has a smaller value than the previous entry. If the entry has a smaller value, then the previous entries have to re-sorted with the entry until either an entry that is smaller is found or the start of the list is reached.
\\[11pt]
The insertion sort is simple to implement, only consisting of two loops. Because of its simple design, it is a sorting algorithm that is quite often used. Insertion sort is however quite slow compared to most other sorting algorithms. It has an $O(n)$ average run time and a runtime of $O(n^2)$ in terms of a worst-case scenario.
\\[11pt]
Insertion sort is quite effective when it is used to sort almost fully sorted lists. This means that a lot of other sorting algorithms uses the insertion sort algorithm. It is usually used as a final step in other sorting algorithms. Examples of this being the sorting algorithms Shell sort, Merge sort and Quick sort.