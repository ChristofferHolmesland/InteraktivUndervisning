\section{Python}
Python\cite{PythonOrg} is a dynamically-typed object oriented programming language developed by The Python Software Foundation. Dynamically-typed means that you don't need to specify the type of variables, because the type is determined at runtime. The syntax is very similar to JavaScript, however Python uses indentation instead of curly brackets to seperate code.
\\[11pt]
Classes are defined using the \code{class} keyword. Functions are defined using the \code{def} keyword. Methods (functions inside classes) are defined as functions inside a class with the first argument begin a reference to the instance of the class. A class constructor can be added by defining a method with the name \code{\_\_init\_\_}. Example of a class:
\begin{lstlisting}
# This is a comment
class Person:
    def __init__(self, first_name, last_name, age):
        self.age = age
        self.setName(first_name, last_name)

    def setName(self, first, last):
        self.name = first + " " + last

p = Person("C", "H", 88)
\end{lstlisting}