\subsection{Binary Search Tree}
A Binary Search Tree (BST) tree is an evolved form of the Binary Tree, that is structured in a way that is more beneficial for storing large amounts of data. The criteria for a Binary tree to be a binary search tree is the following conditions:
\begin{itemize}
    \item{All left children nodes need to have a value lower than its parent node.}
    \item{All right children nodes need to have a value higher than its parent node.}
    \item{There cannot be any nodes with duplicate values in the tree.}
\end{itemize}
Having these requirements helps the BST search for the correct node, since the time it takes to find the correct node is shortened tremendously. Inserting new nodes in the tree is also quite simple, just traverse the tree, where the direction is dependent on the new nodes value. Once a leaf node has been reached, put it as either a left child or a right child based on whether the new node's value is lower or greater than the lead node's value.