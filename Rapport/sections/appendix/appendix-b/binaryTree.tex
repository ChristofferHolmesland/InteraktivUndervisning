\subsection{Binary Tree}
A binary tree data structure is a simple data structure whose structure resembles a tree. A binary tree consists of nodes each having a unique value. Each node has a reference to its parent node and its children nodes. The previous node that links to the current node is called a parent node. The child nodes are the next nodes in line after the currently selected node. A tree node can only have up to two children node, and every node must have only one parent node. The exception to this rule is the very first node in the tree, which is called the root node. There can only be one root node in the tree. Since a node can only have up to two children nodes, the child nodes are usually referred to as either the left child node or the right child node. A node that has no child nodes will always be at the bottom of the tree and are referred to as leaf nodes. The binary tree data structure is notably used for file systems.