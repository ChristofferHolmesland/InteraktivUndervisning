\subsection{Binary Tree}
A binary tree data structure is a simple data structure whose structure resembles a tree. A binary tree consists of nodes each having an unique value. Each node has a reference to its parent node and its children nodes. The previous node that links to the current node is called a parent node. The child nodes are the next nodes in line after the current selected node. A tree node can only have up to 2 children node, and every node must have only 1 parent node. The exception to this rule is the very first node in the tree, which is called the root node. There can only be 1 root node in the tree. Since a node can only have up to 2 children nodes, the child nodes are usually referred to as either the left child node or the right child node. A node that has no child nodes will always be at the bottom of the tree, and are referred as leaf nodes. The binary tree data structure is notably used as file systems.\cite{BinaryTree} 