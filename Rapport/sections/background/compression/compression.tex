\subsection{Compression}
When a user connects to the webpage, one of the things that happens is that a request is sent to the server requesting the html file. If the request is valid it will send back the html file and when it's rendered it will then request .js and .css files when it's needed. This is no problem when the files are small or if you have a good connection, but if the files get larger or the user isn't connected on a fast connection. The user will have a bad experience. This can sometimes be solved by using compression, and in this case all files sent to the client is compressed with the compression package using GZIP compression. As long as the application is divided up and only data that is required is sent, the compression package helps with the filesize on the files that the user request, and in our case it reduced the size of files up to 80\%.