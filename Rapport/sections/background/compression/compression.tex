\subsection{Compression}
When a user connects to the webpage, the browser requests the HTML file from the server. If the request is valid, it will send back the HTML file. When it is rendered, it will request JS files and CSS files when it is needed. This is no problem when the files are small or if the user has a good connection, but if the file size gets larger or the user is not connected on a fast connection. The user will have a bad experience. This can sometimes be solved by using compression, and in this case, all files sent to the client is compressed with the compression package\cite{Compression:Info} using GZIP compression.\cite{GZIP:Info} As long as the application is divided up and only data that is required is sent, the compression package helps with the file size on the files that the user request, and in our case, it reduced the size of files up to 80\%.