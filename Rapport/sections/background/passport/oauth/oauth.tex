\subsubsection{OAuth 2.0}
OAuth 2.0 is the most used authorization protocol today. The OAuth protocol was originally developed for systems to easily and securely get access to user information stored on other systems over the internet.\\
OAuth 2.0 work flow consists of the client, that is the third party system that wants information about the user, first redirects the user to the authorization server together with wanted scope, in order to authorize the user. After the user has been authorized, the user will be asked whether or not he will consent to allow the client access to data requested. What kind of data and what the client can do with them is determined by the type of scope that is sent with the request. If the user consent the user will be redirected back to the client page on a callback url and an authorization code is given to the client. This code is later sent back to the authorization server in order to authorize the access to the user's resource owner. If authorization code is valid an access token is given to the client which the client can use it to access the users resource server and obtain the requested data.\\
Most of the webpage redirects happens on front channels, while obtaining and using the authorization token happens on the back channel. This is done in order to make sure that no sensitive data is accidentally intercepted by someone with malicious intent. It also guarantees that even if someone managed to intercept the authorization code sent from the authorization server ahead of time, they still wouldn't be able to use it on the user's resource server. This is because the resource server requires a high secure tunnel on the server side.