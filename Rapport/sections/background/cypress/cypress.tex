\subsection{Cypress}
The application required some end-to-end tests in order to check the functionality of the website. In this project, Cypress was chosen as the testing tool for these kinds of tests.
\\[11pt]
Cypress\cite{Cypress:MainPage,Cypress:howItWorks} is an open source web testing framework that is primarily used for end-to-end testing. Cypress is designed in a way that allows test-driven development to be used more efficiently in web development. Cypress allows the developer to efficiently write test suites to their website during development in concurrently to coding the application. This makes Cypress scale a lot better compared to most other end-to-end open-source test frameworks.
\\[11pt]
An alternative to Cypress was using a Selenium based framework. Selenium\cite{Selenium:Intro,Selenium:Wiki} is an open-source framework used for automation testing on web applications. Selenium was first released in 2004 and is still currently the most used tool for automation testing for web pages. Selenium is not a single tool; rather the framework is built up by numerous Selenium tools that are all merged. Some of these tools include Selenium IDE, Selenium WebDriver, Selenium RC (Remote Control). Selenium has the advantage over Cypress when it comes to supporting multiple programming languages like Java, Python, and C\#, supporting more web browsers, and having the option to record the user actions and converting it to test scripts. Unfortunately, this comes at the price of having to install and setup each of the Selenium tools.
\\[11pt]
Despite one of the group members having some previous experience using Selenium, Cypress was instead chosen. There were multiple reasons for why Cypress was preferred over Selenium. The first reason was that Cypress only required tests to be written in JavaScript similarly to the Mocha unit test framework, whereas Selenium usually required the use of drivers and conversations between different languages. Since the application was primary developed using JavaScript, the group felt there was not any reason for using Selenium over the fact that it supported multiple languages. The second reason was that Cypress was built to handle modern JavaScript frameworks; this includes supporting Vue, which Selenium seemed to have several well-known issues with. Selenium requires many tools needed to be used together in order to write functional end-to-end tests. Cypress is a tool that has all these tools built into it. This makes Cypress relatively easy to install and get started with. Selenium, on the other hand, required the installation of multiple Selenium licenses tools, libraries, and other testing frameworks to get started. All in all, the advantages of Cypress strongly outweighed the advantages of Selenium for this application.\cite{CypressVsSelenium}