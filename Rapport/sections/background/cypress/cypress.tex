\subsection{Cypress}
The application required some end-to-end tests in order to fully check the functionality of the website. In this project Cypress was chosen as the testing tool for these kinds of tests.
\\[11pt]
Cypress\cite{Cypress:MainPage,Cypress:howItWorks} is an open source web testing framework that is primarly used for end-to-end testing. Cypress is designed in a way that allows test driven development to used more efficiently in web development. Basically, allowing the developer to efficiently write test suites to their website during development in parallel to coding the application. This makes Cypress scale a lot better compared to most other end-to-end open-source test frameworks.
\\[11pt]
An alternative to Cypress\cite{Selenium:Intro,Selenium:Wiki} was using a Selenium based framework. Selenium is an open source framework used for automation testing on web applications. Selenium first released in 2004 and is still currently the most used tool for automation testing for web pages. Selenium is not a single tool, rather the framework is built up by numerous Selenium tools that are all merged together. Some of these tools include Selenium IDE, Selenium WebDriver, Selenium RC(Remote Control), etc. Selenium has the advantage over Cypress when it comes to supporting multiple programming languages like Java, Python and C\#, supporting more web browsers, and having the option to record the user actions and converting it to test scripts. Unfortunately this comes at the price of the job of installation and setup for each of the Selenium tools.
\\[11pt]
Despite one of the group members having some previous experience using Selenium, Cypress was instead chosen. There were multiple reasons for why Cypress was preferred over Selenium. The first reason was that Cypress only required tests to be written in JavaScript in a similar fashion to the Mocha unit test framework, whereas Selenium usually required the use of drivers and conversations between different languages. Since the application was primary developed using only JavaScript, the group felt there was not really any reason to using Selenium over the fact that it supported multiple languages. The second reason was that Cypress was built to handle modern JavaScript frameworks, this includes supporting Vue, which Selenium seemed to have several well-known issues with. There are a lot of tools needed to be used together in order to write functional end-to-end tests. Cypress is a tool that has all these tools built into it. This makes Cypress relatively easy to install and get started with. Selenium on the other hand required the installation of multiple Selenium licences tools, libraries and other testing frameworks to get started. All in all, the advantages of Cypress strongly outweighed the advantages of Selenium for this application.\cite{CypressVsSelenium}
\\[11pt]
The end-to-end tests in this project are stored in the "test" folder, together with the unit-test, but in its own folder with the name cypress. There are detailed instructions on how to install and starting cypress on the projects README file. Running the end-to-end test requires only clicking on any of the files, once the cypress interface is open and visible. The different test files can be found in the with the name "specs". Essentially all describe functions in the test files are counted as their own test suite. It is intended that the end-to-end tests are run using the project's testing mode. In this mode the server will run on port :8082 instead of :8081. It also requires the user to set up an environment file to test. To build the client for testing mode, use the "npm run buildTest" command.
