\subsection{Cypress}
The application required some end-to-end tests in order to fully check the functionality the web-site. In this project Cypress was chosen as the testing tool for these kinds of tests. \\[11pt]
Cypress is an open source web testing framework that is primary used for end-to-end testing. Cypress is designed in a way that allows test driven development to used more efficiently in web development. Basically allowing the developer to efficiently write test suites to their website during development in parallel to coding the application. This especially makes Cypress scale a lot better compared to most other end-to-end open-source test frameworks. \\[11pt]
An alternative to Cypress was using Selenium based framework. Despite one of the group members having some previous experience using Selenium, Cypress was instead chosen. The were multiple reasons for why Cypress was preferred over Selenium. The first reason was that Cypress only required tests to be written in JavaScript in a similar fashion to the mocha unit test framework, where as Selenium usually required the use of drivers. Since the application was primary developed using only JavaScript, the group felt there was not really any reason to using Selenium over the fact that it supports multiple languages. The second reason was that Cypress was built to handle modern JavaScript framework, this includes supporting Vue, which Selenium seemed to have several well known issues with. There are a lot of tools needed to be used together in order to write functional end-to-end tests. Cypress is a tool that has all of these tools built into it. This makes Cypress relatively easy to install and get started with. Selenium on the other hand requires you to install drivers,libraries and testing frameworks separately to the installation of Selenium. All in all the advantages of Cypress strongly outweighed the advantages of Selenium. \\[11pt]
The end-to-end tests in this project are stored in the "test" folder, together with the uni-test, but in its own folder with the name cypress. There are detailed instructions on how to install and starting cypress on the projects README file. Running the end-to-end test requires only clicking on any of the files, once the cypress interface is open and visible. The different test files can be found in the with the name "specs". Essentially all describe functions in the test files are counted as their own test suite. It is intended that the end-to-end tests are run using the testing mode. In this mode the server runs on port :8082 instead of :8081, requires the user to build the project using the "npm run buildTest" command, and set the users environment file to "test". It is recommended by Vue to set data-attributes on the html elements that were going to be tested. This is to make it easier for Cypress to select and obtain the correct html element. This is normally a problem since the html elements has a tendency to change constantly their attributes during development in a Vue application. Unfortunately these could potentially present a security risk if they were present in the production build. Unfortunately a method of removing these in only the production mode, while keeping them in testing mode was never discovered. This is something to keep in mind for the future development of the application. All the data-attributes used for the current tests have been well documented in a txt file and can be found in the same directory with the rest of the end-to-end tests.