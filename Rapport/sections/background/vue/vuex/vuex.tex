\subsubsection{Vuex}
One problem that is common to run into when creating a medium to large size application in Vue, is how to handle variables used by more than one component. This is solved by a plugin called Vuex\cite{VUEX}, which makes it possible to store variables in one location and let all components read and mutate variables. When creating a store, it is possible to split it up into smaller stores and still have all the components be able to access the entire store. A store will mostly consist of a state, mutations, actions, and getters. The state is where all the variables are stored. To retrieve the state getters are used. Getters cannot mutate the state; only retrieve the state. In order to mutate a variable, mutations must be used. This is accomplished in two steps. The first step is to call an action function. This contains the business logic where the function has read access to the state. The action function is usually used for format validation and to call one or multiple mutation functions with the desired state change. The mutation function is the second step, and has write access to the state. The mutation functions are where the state change happens. The getters and actions are accessible from all components in the Vue project. Vuex allows the variables to be mutated in a predictable way, making sure that the state is not changed unexpectedly.