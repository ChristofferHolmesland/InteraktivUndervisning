\subsection{Canvas}
In web development there are three ways to display graphics to the user. HTML elements can be added and removed to the page using JavaScript, and elements can be styled using CSS. This is very slow because manipulating the DOM means that the whole page has to be rendered again. Another option is to use SVG (Scaleable Vector Graphics), which is done by adding elements to the SVG object. When elements are added to the SVG object, they are also added as elements to the webpage. Frequent DOM updates are very slow, because the position of every element has to be recalculated. When visualizing graphs and other datastructures both SVG and Canvas are good options. However, since this project requires the user to interact with the datastructures in realtime, the canvas element was selected. The canvas element exposes a drawing API (Application programming interface), which lets the developer display things on it. Text, lines and other simple shapes can be drawn. The disadvantage of using canvas over SVG is scaling graphics. The user should be able to zoom in and out while modifying a graph. Compared to how this can be achieved with the canvas, the SVG solution is simple.