\subsection{NodeJS}
JavaScript has traditionally been a language which could only be used in web browsers. This means that client code had to be in JavaScript and server code had to be in another language. When working on the same project it is often easier to only use one langauge. NodeJS is a JavaScript runtime which makes it possible to run JavaScript code on both client and server side. Because the code doesn't run in a browser, it's possible to access the file system and the operating system of the machine running the application. NodeJS comes with a command line tool called npm (node package manager) which can be used to install packages in your project. Packages are code which other developers have written and shared. This feature makes it easier to not "reinvent the wheel" when creating your application. There is only one viable alternative to NodeJS, which is Deno. Deno is made by the same person as NodeJs, but is supposed to be more secure. One of ways of achieving this is by removing the npm tool, which is the reason why NodeJS was chosen over Deno.