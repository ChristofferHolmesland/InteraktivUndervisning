\subsection{Localization}
A feature that allows a larger group of people to use the application is allowing the user to change the language used throughout the application. As a base, the application lets the user select between English and Norwegian. The feature is structured in a way such that it allows easy addition of new locales. A locale contains all the text needed for the application for a given location. All locale files are loaded into RAM when the server starts. Locale files for this project are around 30KB. Since every user connecting to the server requests a locale file, it is an advantage to store them in RAM where the server has quick access. The file size allows the server to store them in RAM without having a significant performance impact even if there where a large number of locales. On the client the default locale is Norwegian, and on the navbar, the user can change to a different locale. When the user changes the locale, a request is sent to the server, where the server will find the correct locale and send it back. When the client receives the response it stores the locale inside the Vuex store. This allows all Vue components to access the locale. The structure of a locale file is a JSON object where usually a component will have its name as a property in the object. This allows each component to only get locale that is for itself. This setup also makes the addition of new locales to be very easy. When adding a new locale a new file with the filename of the locale is created, and the content from another locale file is copied over and translated. When the server starts, it reads the content of the folder containing all the locales, so there is no need to add code when implementing new locales.