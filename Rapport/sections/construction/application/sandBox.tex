\subsection{Sandbox}
The application also contains a sandbox area, where students can practice the different question types currently being used. The sandbox was designed to give the user a quick introduction on how the selected question types works. This includes allowing them to try out whatever tool needed to answer the given question type. Therefore, the GraphDrawer component is available to be used for certain questions. The content of the Sandbox.vue file changes drastically to the user depending on the value currently being selected in the vue select box at the top right corner of the page. This select box is mounted to the variable “questionTypes” stored on the Sandbox.vue. Depending on the value given to the questionType variable, the Sandbox.vue content will change in order to accommodate the chosen question type. This is done by using v-if and are written and these follow the same syntax as normal if and else statements in JavaScript. Question type Text is the default value for the select box. The Sandbox view only uses the external GraphDrawer.vue component, no other component was needed here. \\[11pt]
The format used in the Sandbox.vue view on the other hand are the same for the different question types. The available tools that are needed for a question type is always displayed in the middle of the page in its own container. There is a guide section that will display a quick introduction to the question type, followed by the information needed in order to solve this type of question. Some question type will also display a settings section that allows the user to change the default information in the dummy question that is shown. The “getShowSettings” function is responsible for revealing the settings section when it is required, otherwise the function is responsible for removing the section altogether.  The content in settings is determined by the question type and uses the same v-if system as the earlier segments. The guide and settings section are also mounted to their own Boolean variable called “showGuide” and “showSettings”. These variables whether the content in these sections are displayed or hidden to the user. All the text used in the Sandbox.vue is stored in the locale’s files. This was done so that the component could easily support multiple languages. \\[11pt]
As stated earlier, the main content in the SandBox.vue file change depending on what the question type is chosen in the select box. The data function in the Sandbox file contains all the needed data for setting up the playground for the different question types. To start off with, when the question type is “Text”, an input field that is not mounted to any variable is displayed. The purpose here was to allow the user to write in an input field closely resembling the one used in a normal “Text” question. When “MultipleChoice” is chosen, the container will have a b-row containing 4 b-columns where each column is a checkbox that is representing one of the available options. The content of the multiple-choice data is obtained from the data function in the object called “MultipleChoiceChoices”. The question types MergeSort, QuickSort, Tree, Dijkstra and Python are all using the GraphDrawer component, however the settings used on the GraphDrawer varies between the question types. Sorting questions that use the GraphDrawer has their control type set to sort. They also have their respective sort types specified in the “sortType” parameter. The initial data used for the starting array is obtained from the data function from the arrays “mergesortSteps” and “quickSortSteps”. These arrays can be updated by the user using the settings input fields. The input field is mounted to a string in the data function with the same name as the sorting algorithm. Whenever a change to the input field occurs a function is called that splits up the string and creates an array based upon the string value. This array is later used to overwrite the list in the steps array, which changes the displayed array on the site with the updated information. The Tree question will allow the user to practice drawing using the GraphDrawers Graph0 mode. The Dijkstra question lets the user practice showing which route the Dijkstra algorithm will perform. The data needed for the Dijkstra question used in the Sandbox is stored in a separate file called “dijkstraQuestion.js “, which is stored in the sandbox folder. The file is imported and returns an object which given to the data function so that the GraphDrawer can easily link to it. The GraphDrawer will be using the Dijkstra control type for these types of questions. Python questions follow the same GraphDrawer functionality as Dijkstra questions, but it also requires a text field where the student can write its python code. The python questions are set up so that the student can answer the given python question using the GraphDrawer. The GraphDrawer will be able to handle unique classes if the student initializes it in the textfield and clicks on the parse button. This essentially lets the student the option of creating their own python related questions. The last question type Shell sort uses the same settings functionality as the other sorting question algorithms, in order to change the initial list and initial k value. The container contains only a b-row and a b-column, where all the list items are disabled input fields and are placed inside the b-column. The sandbox for Shell sort questions are designed to let the student create and solve shell sort questions with functionality akin to the ones used in a normal session. Essentially lets the user add new list on a new row, change the order of the list items in new row by clicking on two different list items and have the option to delete lists as long as its not the initial list.

