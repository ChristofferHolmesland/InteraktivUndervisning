\subsection{Active session}
For an admin or a student assistant, there will be a selector on the main admin dashboard for selecting a session. If there is no sessions, the user will be promted to create one. After the user has selected a session the start button will be enabled. Pressing this button will send a message to the server requesting this session to be started. The server will the proceed to request information from the database of the session. Information of the session will be stored in an object in a map. After the server has made the session ready, it will send a message back to the client and switch the view to a waiting room. Here it will show a code that students can use to connect and also display the current number of student connected. 
\\[11pt]
On the main client dashboard there will be an option to join an active session. Once a client have joined a session, the user will also join a room in socket.io. This is a feature that allows the server to send a message to all clients connected to that room. This is used when the session is going from the waiting room to a question, or to the next question. When the admin send a start signal, the server will gather information that it needs for the next question and send a message to all clients connected to that room.
\\[11pt]
When a student recieves a message with a question, it will switch to a component made to display a question. There is also a similar component for the admin, but they will differ some as they need other features. When the question component for a student is first shown, it will display the answer tab, but the student can switch to see the question. This was done in an effort to give all students the possibility to view each question, if the display in front of the class was hard to see. When a student is on the answer tab, there will be a component allowing the student to give their answer in a way that is specific for that question type. When the student is ready to answer, they will have to press either the button to send in what they answered or press the button to answer that they didn't know.
\\[11pt]
When a user sends in their answer it will be sent to the server, where it will go through a solution checker. When using the solution checker, the main script will need the answer, solution and question type. When getting to the main script, it will look at the question type and the send the answer and solution to the correct checker. There will be a solution checker for each question type, and it will go through the answer and compare it with a solution. The comparrison will depend on the question type, but in general it will loop through the solution and compare the current loop step with whats in the answer. The current solution checker structure can be seen in figure \ref{fig:serverStructure}, where the main script is called "Solution generator" and then all the sub scripts are for each question type. This not only makes it easy to expand with new question types, but also makes the debug for solution checking easier.
\\[11pt]
When the server is finished with the solution checker, it sends a message back to the student moving them to a waiting screen with the result. The result will be shown as a text and will be as discreet as possible. While the student is move to the waiting area, the server will then store the answer in the database and then send a message to the host of the session informing how many users have answered the question.
\\[11pt]
For the session to move on from a question there are three ways. If all users connected to the session has answered the question it will move on, if the timer runs out or if the host presses the next button. All students will still be in the waiting area, while the server will send sats and all the wrong answers with the solution to the host. The admin will then be shown basic stats and a list of all wrong answers, where the host will be able to compare the wrong answer with the solution. When the host is reade for the next question there will be a button to press to move on. This will send a message to the server requeting the next question, and if there is a new question it will be sent out to all students connected to the room and the host. When there are no question left, an end screen will be shown for both the students and the host informing them that the session is now over and giving them a button to return to the dashboard.
\\[11pt]
During a session users will be able to leave and join the session. To leave a session, they can either press the leave session button, promting comfirm dialog, or just leaving the web site. This will send a message to the server requestiong the user to leave the session. This will trigger a number of things to happen. The server will first check if the user have answered the question or not. The server uses different counters to keep track of users during a session. This is done to make sure the counter on the host screen shows the correct amount of answers and users. It is also used to go to the result screen when all users have answered. If the host loses connection, refreshes the page or leaves the page a counter will start and gives the host 15 minutes to reconnect. No student will be kicked out, and will be able to answer on the current question. If the host don't reconnect the session will be removed as an active session, and all students will be moved to the end screen.