\subsection{Sorting}
Sorting algorithms are used to reorder elements in a data structure, often arrays or lists. The elements are ordered by a rule depending on what type of element the data structure contains. Numbers are often ordered from the lowest number to the largest. Characters are often ordered by transforming the character into a number, where "a" is often defined as smaller than "b."
\\[11pt]
To sort the elements an algorithm must be used. They have different performance and memory usage, so it is important to be able to pick the best fitting algorithm depending on the situation. This project contains implementations of insertion sort, shell sort, merge sort and quick sort which are common algorithms. They are written to work on arrays containing any element which can be compared using the $>$, $<$ and == operators in JavaScript. Instead of returning the sorted version of an array, they return functions which can be called to navigate between the steps of the algorithm. When checking if a student's answer is correct, it is pointless to check if they managed to sort the array, because they might not have sorted the array according to the rules of the algorithm. The steps are therefore returned so that together with the GraphDrawer, the students can graphically perform the sort. It is then possible to compare each step with what the student did and see if they know how the algorithm works.

\subsubsection{Merge Sort}
While performing the sort, every stage of the algorithm is saved in a list of steps. There are three stages which are stored as a step.
\begin{itemize}
    \item Initial. The initial step is stored before the sorting starts. It contains a copy of the unsorted array and a reference to the limit.
    \item Split. The split step is added when an array is split into two arrays. It contains a copy of the array being split, and copies of the new arrays.
    \item Merge. The merge step is added after two arrays have been merged to a new array containing the sorted version of all the elements. It contains a copy of the two arrays being merged and a copy of the resulting array.
\end{itemize}
When implementing the algorithm, performance and memory usage were not considered to be important. The algorithm runs only once per question to generate the steps so that each student's answer can be compared to the right way of performing a merge sort. For this reason, the intermediate arrays are allocated dynamically instead of being reused in later steps. This gives the algorithm worse performance, but its average case performance is still $O(n * logn)$. Memory usage has not been considered because the algorithm has to store more information than normal to store the steps.

\subsubsection{Quick Sort}
While performing the sort the algorithm, it also stores the state in steps. This is done for the solution checker to be able to check if the student has drawn the correct answer when simulating the quicksort algorithm.
\begin{itemize}
    \item Initial: The initial step is stored before the sorting starts. It contains a copy of the unsorted array.
    \item Split: The split step is stored each time the algorithm splits a list; it stores the unsorted list, the pivot point, left and right list.
    \item Merge: The merge step is stored each time the algorithm merges the sorted lists and pivot point. It stores information about the sorted left and right list, the pivot point and the sorted list after the merge.
\end{itemize}
\begin{figure}[H]
    \includegraphics[width=\linewidth]{/diagrammer/Quicksort}
    \caption{Quicksort}
    \label{fig:quicksort}
\end{figure}
