\subsubsection{Vue}
When creating this project we knew that a framework was needed for both the server and client. For the client, a few frameworks were looked into, such as Vue, Angular\cite{Angular:Info}, React\cite{React:Info}, etc. In the end, Vue was picked as it serves our needs better in both functionality and how easy it was to learn. When creating a Vue project, components are created in Vue files. Vue files can include many elements, but in general, it includes HTML, JavaScript code, and CSS. When building a project, the Vue CLI tool generates a single HTML file, JavaScript files and CSS files to a configured build directory. This directory is put inside the public folder on the server. This made it so that when changes are made on the client files it would rebuild and restart the server taking advantage of hot updating the web site while developing. The Vue file is divided into different parts by using tags to let Vue know the difference between HTML, JavaScript, and CSS.
% TODO Include image of an empty vue file with only tags for template, script and style
\\[11pt]
One of the benefits of using Vue is that the application can be divided into smaller files. If code is reused it is possible to create a single file and import it where it is needed. Another benefit of using Vue is that it is possible to watch a variable and re-render parts of the page if it changes. Dividing up the code into smaller files also makes it easier to debug since the code that needs to be looked at is smaller.
% TODO insert full tree of file structure on client
\\[11pt]
Every Vue component uses different features from the framework. The most used features for each component are prop, data, methods, computed and watch. Some of the more rare features are created, mounted and beforeDestroy. Each of these are part of making the web page responsive and reactive when variables changes. Created is a function that runs when the component is first loaded into the DOM, mounted runs just before the HTML in the component is rendered, and beforeDestroy runs before the component is removed from the DOM. The variables that are used in a component can be stored in two places, either in props or data, but they have two different use cases. If the component needs to get a variable from their parent component it should be stored as a props variable. A props variable cannot be changed from the component. If the parent component changes the value, the child component will get the new value. If the value is used in the HTML, the component is rendered again with the updated value. If the value needs to change inside the component the value needs to be stored inside a data property. This variable also triggers updates to the renderer if it is changed, but it has no link with the parent component. Methods are normal functions that can be called by other functions or by HTML elements. Computed methods run once the component is created, and every time a variable inside the function is changed. Watch methods watch a variable or a computed method and trigger a method every time it changes, where the new and old value is passed as arguments to the method.
% TODO add picture with an example of a vue file