\subsubsection{Vue}
When creating this project we knew that a framwork was needed for both the server and client. For the client, a few frameworks were looked into, such as Vue, Angular\cite{Angular:Info}, React\cite{React:Info}, etc. In the end Vue was picked as it serves our need better in both functionality and how easy it was to learn. When creating a vue project you write in .vue files. Where you will have a few options on what it includes, but in general it can include HTML code, JavaScript code and CSS code. When building a project, it will generate a single HTML file, JavaScript files and CSS files to a configured build directory. This directory was put inside the public folder on the server. This made it so that when doing changes on the client files it would rebuild and restart the server taking advatage of hot updating the web site while developing. The vue file would be divided into different parts by using tags to let vue know the difference between HTML, JavaScript and CSS.
% TODO Include image of an empty vue file with only tags for template, script and style
\\[11pt]
One of the benefits of using vue is that you can break up your application into smaller files, and if you use some form of code more then once you can create a single file and import it where its needed. Another benefit from using vue is that you will be able to watch a variable and re-render parts of the page if it changes. Breaking up the code into smaller files also makes it easier to debug since the code that you need to look through gets smaller and it will not include code that have nothing to do with the code you are debugging.
% TODO insert full tree of file structure on client
\\[11pt]
Every Vue component uses different features from the framework. The most used features for each component are prop, data, methods, computed and watch. Some more rare features are created, mounted and beforeDestroy. Each of these are part of making the web page responsive and reactive when variables changes. Created is a function that will run when the component is first loaded into the DOM, mounted will run just before the HTML in the component is rendered and beforeDestroy is run before the component is removed from the DOM. The variables that is used in a component can be stored in two places, either in props or data, but they have two different use cases. If the component needs to get a variable from their parent component it will be stored as a props variable. A props variable can not be changed from the component, but the parent component that sent it can change it and will be changed in the child component. If the value is used in the HTML, the component will be rendered again to use the updated value. If you want to change the value from its own component the value needs to be stored inside a data property. This variable will also trigger updates to the renderer if it's changed, but it will have no link with parent components. Methods will be normal methods that can be called by othe methods or by for example buttons. Computed methods will run once when the component is created and each time a variable inside the function is changed. Watch methods will watch a variable and trigger a method each time the variable changes, where you can have arguments of the new and old variable.
% TODO add picture with an example of a vue file