\subsubsection{Vue Router}
Since vue is a single page application it will not scale well since it would have to load in all resources even if it didn't use them. A solution to this is vue routing where it will still be a single page application, but it will simulate the experience of a multi page application where only the resources needed for what is shown is loaded. When the user goes to a new route the JavaScript and CSS files needed will be loaded. In combination with the vue cli it is a very easy setup, and easy to use. If vue routing is selected when setting up the project it will create a views folder and the config file for vue router. When a new route is desired the only thing that is required is that the developer creates a new .vue file and uses it as a normal component and that there is created a new link in the config file for vue router linking back to that .vue file. The standard is to place all routes or views inside the views folder and all components that are used by views inside the components. Even though they are both .vue files and you could tell the difference if you looked inside them. Each route will create its own JavaScript file when you build the project and is only loaded in for the user, when the its needed.
% TODO include image showing how a include of a view is done in routing config file.