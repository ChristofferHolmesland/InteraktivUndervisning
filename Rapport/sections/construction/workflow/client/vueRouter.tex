\subsubsection{Vue Router}
Since Vue is a single page application, it does not scale well since it would have to load in all resources even if it did not use them. A solution to this is Vue routing where it is still a single page application, but it will simulate the experience of a multi page application where only the resources needed for the shown content is loaded. When the user goes to a new route the JavaScript and CSS files needed are loaded. In combination with the Vue CLI tool it is straightforward to set up, and easy to use. If Vue routing is selected when setting up the project, the tool creates a views folder and the config file for the Vue Router. When a new route is desired the only thing that is required is that the developer creates a new Vue file and uses it as a normal component. A new link needs to be created in the Vue Router config file pointing to the Vue component. The standard is to place all routes and views inside the views folder. All the components used by the views should be placed in a separate folder. Even though they are both Vue files and it is not possible to tell the difference by looking at the content inside the files.
For every route, a separate JavaScript file is created which is only loaded when the user visits that route.
% TODO include image showing how a include of a view is done in routing config file.