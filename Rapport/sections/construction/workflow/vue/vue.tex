\subsubsection{Vue}
When creating this project we knew that a framwork was needed for both the server and client. On the client we looked at a few different option such as VueJs, Angular, React, etc. but in the end Vue was picked as it server our need better in both functionality and how easy it was to learn. When creating a vue project you write in .vue files. Where you will have a few options on what it includes, but in general it can include HTML code, JavaScript code and CSS code. When you build the the project it will generate a single HTML file, JavaScript files and CSS files to a configured build directory. We put that directory inside the public folder on our server. This made it so that when doing changes on the client files it would rebuild and restart the server taking advatage of hot updating the web site while developing. The vue file would be divided into different parts by using tags to let vue know the difference between HTML, JavaScript and CSS.
% TODO Include image of an empty vue file with only tags for template, script and style
\\[11pt]
One of the benefits of using vue is that you can break up your application into smaller files, and if you use some form of code more then once you can create a single file and import it where its needed. Another benefit from using vue is that you will be able to watch a variable and re-render parts of the page if it changes. Breaking up the code into smaller files also makes it easier to debug since the code that you need to look through gets smaller and it will not include code that have nothing to do with the code you are debugging.
% TODO insert full tree of file structure on client
\\[11pt]
Inside each vue file in the script tag we use different included features from vue. Those that we use for almost each component is prop, data, methods, computed and watch. Some more rare features are created, mounted and beforeDestroy. Each of these are part of making the web page responsive and reactive when variables changes. Created is a function that will run when the component is first loaded into the DOM, 