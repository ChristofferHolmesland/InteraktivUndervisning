\subsection{Localization}
One feature that allows a larger group of people to use the application, is allowing the user to change the language used throughout the application. As a base the application lets the user select between english and norwegian. The structure of the feature allows easy addition of new languages. All locale files are loaded into RAM when the server starts. Locale files for this project is around 30KB. Since every user connecting to the server requests a locale file, it's an advantage to store them in RAM where the server has quick access, and the file size allows the server to store them in RAM without having a major impact even if there where a large amount of locales. On the client the default locale is norwegian and on the navbar the user can change to a different locale. When the user changes locale a request is sent to the sever, where the server will find the correct locale and sent it back. On the response the client will store the locale inside the vuex store. This will allow all vue components to access the locale. The structure of a locale file is an object where usually a component will have it's own name as a property name in the object. This allows each component to get only locale that is for it self. This setup also makes the addition of new locales to be very easy. When adding a new locale a new file with the filename of the locale is created, and the content from another locale file is copied over and translated. 