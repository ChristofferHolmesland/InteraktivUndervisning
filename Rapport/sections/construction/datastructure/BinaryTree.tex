\subsubsection{BinaryTree}
A binary tree data structure is a simple data structure whose structure resembles a tree. A binary tree consists of nodes each having an unique value. Each node has a reference to its parent node and its children nodes. The previous node that links to the current node is called a parent node. The child nodes are the next nodes in line after the current selected node. A tree node can only have up to 2 children node, and every node must have only 1 parent node. The exception to this rule is the very first node in the tree, which is called the root node. There can only be 1 root node in the tree. Since a node can only have up to 2 children nodes, the child nodes are usually referred to as either the left child node or the right child node. A node that has no child nodes will always be at the bottom of the tree, and are referred as leaf nodes. The binary tree data structure is notably used as file systems. 
\\[11pt]
Most of the student tasks involving binary tree, usually involves drawing a resulting binary tree. In order to avoid giving the students help with drawing a correct binary tree structure, the graph drawer was designed in a way that lets them mix between drawing tree structures and graph structures. However, this essentially meant that multiple extra criterias  needed to be considered when checking the students work with the solution. It was necessary to implement a Javascript class Tree for representing the binary trees drawn and a Binary Node class representing every node in the tree. These classes were designed in a way that they could be used in all the taught binary tree structured in the Algorithms and Data Structures course. This being the normal binary tree structures, binary search tree structures and AVL binary tree structures. 
\\[11pt]
The Tree class consists of a root node referencing the node at the start of the tree, and an array of all the current nodes in the tree, this also includes the root node. The Binary Node class consists of the node's value and an array containing the node's child nodes. A notable design choice chosen was that each binary node add all their children nodes was stored in an array. Normally a tree structured would have only needed having an independent variable referencing the two different child nodes. The reason this was because a student had the option to draw a tree with more than 2 children, which is not allowed in a binary tree structure. This also doesn't cause any problems distinguishing the different child nodes either since the nodes will be divided properly using different placement indexes. Left child will always be located at index 0 and the right child node will always be located at index 1. If there are more than 2 children nodes in the array, then the binary tree criterias are not fulfilled, and therefore will not be accepted as a valid tree object. Multiple functions were implemented to not only transform the drawn tree to a Javascript tree object, but also to check whether or not the created Javascript object is qualified as the chosen binary tree structure. 
