\subsubsection{Camera}
The camera object is responsible for letting the user view only a section of the world at a time. It is implemented in a way that only works for two dimensions. It is also not possible to rotate the camera. This is done to prevent unnecessary complex mathemathical expressions from being used. The camera object has a position given by the \code{centerX} and \code{centerY} properties. Depending on the canvas size and the \code{zoomLevel} a section of the world will be rendered. To determine if a node or edge should be drawn, the \code{cull(object, isNode)} method is used. It returns \code{false} if the object is inside the camera view, or \code{true} if the object is outside. An culled object should not be drawn. An node is culled if it is outside the camera view. An edge is culled if both nodes are outside the camera view.
\\[11pt]
The camera uses the \code{project(x, y)} function to convert screen coordinates to world coordinates. The camera has a position in the world, and also has access to the size of the canvas where user interaction happens. The conversion can therefore be done by checking what percentage of the screen width the interaction happend at, and then adding the same percentage of the camera size along that axis to the position of the camera. This works for both axises. If the camera is positioned with its left side at x=50, and the user clicks in the middle (50\%) of the canvas, then the world position would be \code{50 + camera\_width * 0.5}. To convert from world to screen coordinates, the \code{unproject(x, y)} function should be used. This uses the same method for converting, but in the opposite direction.
% TODO: Insert image showing the relationship between world, camera and canvas coordinates. %
% Insert image showing culling. %