\section{Conclusion}
Content needed in a conclusion:
- What have we achieved, present the finished product
- What have we learned(primary tools: socketIO, vue, cypress, canvas etc)
-Things that have caused problems, and might have slowed down the project process.
-What would we have done differently if we had the chance.
\\[11pt]
In conclusion the current application is a web application that gives teachers the ability to create sessions which contains questions that are relevant to a lecture. The application supports the use of sockets and can have multiple students participate in a session, with no interaction between the participants.  The sessions have layout that support both mobile devices and computers. While the admin portion of the application is strictly designed for pc users only. The application has a drawing tool that is specifically designed to be used by students in order to solve questions revolving certain algorithms and data structures in the courses DAT110 and DAT200. The application will store the results of every question in every session. This feedback can be used by both the teacher and the students. These were the primary goals for the project, and all of them have been completed. Other features such as supporting OAUTH with Open ID Connect for student authorization and multi-language support were added during the development because these features worked well with the planned structure of the application. \\[11pt]
During the development of the application a lot of different tools had to be used. In general, most of tools were to tools that the group had very little to no experience working with. This includes developing single page web applications using Vue, using socket.IO for the WebSocket functionality of the sessions and OAuth and OpenID Connect in order to authenticate students using their Feide users etc. It is safe to say that the by the end of this project that the group members have gained some experience working with these tools. The group has in general learned a lot more about modern web tools and frameworks. \\[11pt]
As stated earlier, the number of algorithms and data structures question types the application currently support is a lot less than originally planned for. This was mostly because we originally underestimated the amount time it required to implement some of new functionalities, and during the last few months of development it was decided to instead prioritize polishing the functionality already available. For instance, implementing the tree related question types and allowing the teacher to insert images to a question, were both implemented tasks that ended up taking a lot longer then intended. Another factor that also slowed the development of application was the amount of time that had to be spent on fixing numerous bugs. The larger the application grew, the more frequently the bugs appeared. It also didn’t help that the GraphDrawer tool had issues at first working with the vue components. The bugs were a constant problem, and there was usually at least one member of the group having to spend the time on fixing known issues during the development of the application. The number of bugs may have been significantly been reduced if the group was more thorough during testing and writing more unit and end-to-end tests, but it can be argued that this would also have slowed down the development process. During the development, the database structure for the application was changed a couple of times. Despite having spent a decent amount of time during the planning process on its design. This was because the group had overlooked multiple features on the site that did not work with the current database structure, and therefore the database had to be changed to accommodate for the missing functionality. This would have been avoided if the group had a designed a better base structure for the database. However, it is not always easy to predict issues with the application. Normally you realize the problem when you encounter it, and by then its already too late.   \\[11pt]
\textbf{What would you have done differently if you had the chance.}