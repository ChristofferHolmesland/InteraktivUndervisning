\section{Conclusion}
In conclusion the current application is a web application that gives teachers the ability to create sessions which contain questions that are relevant to a lecture. The application implements all the features from the primary goals for the project. The application supports the use of websockets and can have multiple students participate in a session. The sessions have layout that support both mobile devices and computers, while the admin portion of the application is strictly designed for pc users only. The application has a drawing tool, that is designed to be used both to answer questions and create questions about certain algorithms and data structures in the courses DAT110 and DAT200. The application stores the result of questions and sessions. This data can be used by the teacher to check if students understand the material in the course. Other features such as supporting OpenID Connect for student authorization, student feedback, and localization support were added during the development because these features worked well with the planned structure of the application. 
\\[11pt]
During the development of the application a lot of different tools had to be used. In general, most of the tools were tools that the group had very little to no experience working with. This includes developing single page web applications using Vue, using Socket.IO for the WebSocket functionality of the sessions, and OAuth in order to authenticate students using their Feide users etc. By the end of this project, the group members have gained experience working with these tools, and the group has in general learned a lot more about modern web tools and frameworks. The group has learned a lot about how to structure larger software projects, and working in teams.
\\[11pt]
One of the challenges with implementing solution checking and generating is that user input can create objects which does not fit with the actual data structure. When answering a question about binary trees, they can create nodes which have more than two children, they can create trees with several roots, or create cycles in the tree. This made it a lot harder to create correct algorithms because there were a lot of edge cases for each question type.  
\\[11pt]

\section{Future Development}
%This is the most important part of the project.

\subsection{Server side}
\subsubsection{Server}

\subsubsection{Database}

\subsection{Client}

\subsection{Login}
When a user vists the website they need to decide between logging in using their Feide account, or continue as an anonymous user. At this stage their user-rights level is 0, and they can only access the login page. Anonymous users have a user-rights level of 1, and are able to participate in sessions. If the users authenticates with their Feide account, they are assigned a user-rights level depending on their role in a given course. A normal user with no special rights will get their level set to 2. Level 3 is used for student assistants. Admins (lecturers, professors, ...) are assigned level 4. Users with a user-rights level above 1 are able to see statistics about sessions they have participated in. Student assistants (level 4) can.
% TODO: Add information about what student assisstants can do %.
Admins (level 4) can create courses, sessions and questions.
\subsubsection{Implementation}
If a user wants to be anonymous, their user-rights level is first set to 1. They are then redirected to the client page.
\\[11pt]
If the user clicks on the Feide login button, a HTTP POST request is sent to /login/feide. When the server recieves the request, it is passed on to PassportJS's authenticate function. The authenticate function redirects the user to an external site for authentication, before redirecting them back to the specificed callback URL on our site. The authenticate function takes an argument telling it where and how to redirect the user, this is called a strategy. The /login/feide route uses the "passport-openid-connect" strategy to connect to UNINETT's Dataporten authentication servers. If the user successfully login with their Feide account they are redirected back to /login/callback/feide. This route uses the PassportJS authenticate function to exchange the access code with an access token which is then passed on to our route handler. The handler reads the HTTP request to get information about their Feide account. This information is used to check the database and create a in-memory user object which the server uses to decide what the user is allowed to do on the server. The user is finally redirected to the /client route where they can join a session.

\subsubsection{Joining a Session}

\begin{figure}[H]
	\begin{subfigure}{0.70\linewidth}
		\includegraphics[width=\linewidth]{userManual/joinSession}
		\caption{}
		\label{fig:joinSession}
	\end{subfigure}
	\begin{subfigure}{0.70\linewidth}
		\includegraphics[width=\linewidth]{userManual/waitingRoom}
		\caption{}
		\label{fig:waitingRoom}
	\end{subfigure}
\end{figure}

\begin{userManualItemlist}
	\item[Step I.] Navigate to the dashboard page.
	\item[Step II.] Click the text area within the "Quick join session" area. (1) (Figure: \ref{fig:joinSession})
	\item[Step III.] Write the session code. A session code is 4 characters long. (Figure: \ref{fig:joinSession})
	\item[Step IV.] Click the button (2) labeled "Join" (Figure: \ref{fig:joinSession}) 
	\item[Step V.] If the session exist, you are send to the sessions waiting room. (Figure: \ref{fig:waitingRoom})
\end{userManualItemlist}

\subsubsection{User Profile}
\begin{figure}[H]
	\begin{subfigure}{0.80\linewidth}
		\includegraphics[width=\linewidth]{userManual/client/userProfile/accessUserProfile}
		\caption{}
		\label{fig:accessUserProfile}	
	\end{subfigure}
	\begin{subfigure}{0.80\linewidth}
		\includegraphics[width=\linewidth]{userManual/client/userProfile/profile}
		\caption{}
		\label{fig:profilePage}
	\end{subfigure}
\end{figure}

\begin{userManualItemlist}
	\item[Step I.] You need to be signed in to your Feide Account to access your user account. 
	\item[Step II.] Click the button (1) with your Feide name on the navigation bar. (Figure: \ref{fig:accessUserProfile})
	\item[Step III.] Click "Profile" button (2). (Figure: \ref{fig:accessUserProfile})
	\item[Step IV.] View user info. The user info contains the Feide username and Feide name. (Figure: \ref{fig:profilePage})
	\item[Step V.] Click the button (3) labeled "Delete my data" if you want to delete your user data on the application. This includes your old session data. (Figure: \ref{fig:profilePage})
	\item[Step VI.] View user stats. The stats includes your total number of sessions, the total amount of questions you answered correctly, the total amount of questions you answered incorrectly and how many times you answered "I don't know". (Figure: \ref{fig:profilePage})
	\item[Step VII.] Choose the course you want to view session data for (4). If you participated in a session corresponding to the selected course, a list of sessions will appear. (Figure: \ref{fig:profilePage})
	\item[Step VIII.] Click on the selected session you want to view your answers for.
\end{userManualItemlist}

\subsubsection{Sandbox}
\begin{figure}[H]
	\begin{subfigure}{0.60\linewidth}
		\includegraphics[width=\linewidth]{userManual/client/sandbox/goToSandbox}
		\caption{}
		\label{fig:goToSanbox}
	\end{subfigure}
\end{figure}

\begin{userManualItemlist}
	\item[Step I.] Navigate to the dashboard page.
	\item[Step II.] Click the button(1) labeled "Go" within the "Go to sandbox" area. (Figure: \ref{fig:goToSanbox})
	
\end{userManualItemlist}


\subsubsection{Localization}
When creating the locale files, it was designed in such a way that every Vue component has its own object. This has been a design that worked well, but during development, a design flaw was discovered. Some Vue components use the same text. This can, for instance, be seen on buttons and error messages. It could have been solved by not just storing locales divided into Vue components, but by also storing common locale in a separate object. Even though the file size for each locale file is relatively small, it can be reduced more. Another way to expand localization would be to add more locales.