\subsubsection{Testing}
\paragraph{Unit-tests}
At the current state, the application does not have nearly as many unit-tests as it could have had. Due to limited time and resources writing unit tests became more of an afterthought in comparison to getting the server functionality working. The unit tests that the project currently has are unit tests for the different algorithms and data structures. These tests have been primarily used in order check that the algorithm implementations work as they should. Areas in the application where there should be unit tests, but currently does not have any are in the Validation Checker and the Solution Checker residing on the server. In the future it might be beneficial if some tests were made for each question type for both checkers. This would in turn help during debugging if any changes were to affect the checkers when implementing a new type of question.
\paragraph{End-To-End tests}
It is recommended by Cypress to set data-attributes on the HTML elements that are going to be tested. This is to make it easier for Cypress to select and obtain the correct HTML element. This is normally a problem since the HTML elements tend to constantly change their attributes during development in a Vue application. Unfortunately, these data-attributes could potentially present a security risk if they were present in the production build. However, a method of removing them only in the production mode, while keeping them in testing mode was never discovered. This is something to keep in mind for the future development of the application. All the data-attributes used for the current tests have been well documented in a text file and can be found in the same directory as the rest of the end-to-end tests.\cite{Cypress:BestPractise}
\\[11pt] 
The future developers should also be aware that Cypress is intended to be used during development following closely the principles of test-driven development. This means that the test themselves need to be altered a lot to handle the changes done to the HTML elements on the site. Writing E2E-tests in Cypress is therefore recommended to be done early in development and not in the middle or at the end of the project.

