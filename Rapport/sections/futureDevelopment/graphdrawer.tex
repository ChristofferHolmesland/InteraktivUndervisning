\subsection{GraphDrawer}
If new question types are implemented which require the GraphDrawer to keep track of thousands of nodes at the same time, there might be performance issues on some systems. The nodes are currently stored in an array. Which means that if you want to find a node on the screen, all of the nodes need to be checked, even the those on the opposite side of the world. Many operations which start on one node, and then end at a node close by the original node (e.g. selecting nodes by dragging the mouse) would be more performant if a spatial data structure\cite{SpatialDatastructure} was used instead.
\\[11pt]
The current system is designed to work with both touch screens and cursors. Because of this, some choices were made to make the development simpler and less time consuming.
\begin{enumerate}
    \item When a user needs to enter text or numbers, the JavaScript alert popup is used because this works in all browsers. Some users with a keyboard would have been able to write faster if the GraphDrawer read the input directly. Mobile browsers does not allow a website to open the virtual keyboard from a script. If the mobile browsers change this in the future, a different method for retrieving the user input could be used.
    \item For simplicitiy both desktop and mobile users control the GraphDrawer the same way. User testing can be done to find the gestures which feel the most natural. If the natural way to do something is different depending on the system, different handlers can be implemented.
\end{enumerate}
Every time something in the world changes, the canvas is redrawn. Changes outside the camera view shouldn't make the GraphDrawer redraw the world. If only a small part of the world inside the camera view is changed, only that part needs to be redrawn. This hasn't been implemented because there haven't been any performance issues with the current question types.