\subsubsection{Server side}
\paragraph{Server}
Due to limited time and resources a decision was made to make a single server and make it responsible for handeling all traffic from the users. This will not be a problem for the server load for the use case at the moment. If the web app should be scaled up to include subjects for the entire university and all its student, or scaled up even further. The server should be split up into microservices, where each service is its own server serving a single purpose. For example everything that has to do with login is its own server and everything that has to do with running an active session is on its own server. These microservices could be placed inside their own docker container and with the use of kuberenetes the servers would scale up and down depending on the amount of traffic. Since the amount of work required in R\&D for all the other features and techologies we used, this was something we would want to do, but simply just didn't have enough time to get done.
\\[11pt]
Another factor that would limit the scalability for the server is the use of a map for connected users and the map for active sessions. After some testing this will not be a problem where the use case is a couple of classes and a couple of active session at the same time. If the traffic where to be scaled up for the entire university or for multiple universities, there could be a problem since the limiting factor would be the available ram on for the server. A way to avoid this would be to redesign how sessions and users are stored in the database, allowing that information that is stored in ram now to be stored directly in the database. 
\paragraph{Database}
Another feature that would suffer if the web app scaled up is the database. The database is using SQLite3 where the limit is 50000 requests per second\cite{SQLite:faqQ19}. This could be a limit if the traffic is scaled up enough, and there should be looked into using another type of database that uses a server to handle the requests. This would also allow the database to be split up into smaller databases and not a single schema for the entire application. Currently the ids are in incremental integers. In the future it would be beneficial to change these to GUID or UUID.\cite{UUIDE}